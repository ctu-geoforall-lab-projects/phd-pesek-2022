\chapter{Conclusion}
\label{conclusion}

The goal of this professional debate was to sketch future directions of the planned doctoral thesis and present its research part.

The first part of the debate was dedicated to the research on the current situation when it comes to \zk{CNN}s usage for classification in the field of remote sensing. This research showed that although there is plenty of articles whose authors like to claim that their models outperform everything else that had ever appeared in the field, such claims very often miss sufficient comparisons with other approaches and architectures to prove them. This research is followed by a study of reviews covering the intended task. It was found that the amount of complex and really systematic reviews dealing with the \zk{CNN}-based classification in the area of remote sensing is sadly pitiable. This chapter tells us that such research is needed and could serve for future applications of \zk{CNN}s in the field as a valuable guide for researchers' approach choose and could stir up the common approach of experiments with the newest architectures without any knowledge about their real performance on satellite and aerial images.

The second part of the debate presents one of the use cases chosen to serve as test applications of various \zk{CNN} architectures. It is a combination of a common task of urban scenery land use classification and the task of differentiation between urban and rural vegetation, a task so far uncovered in the field. For the future doctoral thesis, at least three use cases are intended to be presented, the second one being most probably the detection of horizontal traffic signs.

The ambition of this debate and later the thesis is not to enrich the field of deep learning - such efforts are currently filling seats of research centres all over the world. The ambition of this debate and later the thesis is to finally create a sufficient scientific mycelium on which researchers in the field can rely on and build their future studies on.