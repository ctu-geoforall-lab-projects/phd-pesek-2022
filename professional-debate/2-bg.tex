\chapter{Motivation}
\label{motivation}

In August 2020, an experiment was conducted - a query for papers with the below-described restrictions was done on selected academic database websites. The goal of the experiment was to research whether authors of papers dealing with the use of convolutional neural networks (\zk{CNN}) in the field of remote sensing are comparing results of proposed or used architectures with results of other architectures on the same task. This experiment is summarized in chapter \ref{top-papers}.

This experiment was immediately followed by an overview of the top \textit{review-type} papers - as something akin is the goal of this thesis, it could be taken as the~summary of the~current situation and research on work dealing with the~same problems. This is summarized in chapter \ref{situation}.

% possible titles:
% Top cited papers dealing with use of convolutional neural networks in the field of remote sensing
\section{Articles on CNNs in the field of remote sensing}
\label{top-papers}

\subsection{Web of Science}
\label{wos-papers}

\subsubsection{Using the search term \textit{convolutional neural network}}
\label{wos-papers-full-length}

The first used website was Web of Science\footnote{\url{www.webofknowledge.com}} (\zk{WoS}). The~following query restrictions were used:

\begin{itemize}
	\item \verb|Search string: convolutional neural network|. To focus only on articles dealing with \zk{CNN}s.
	\item \verb|Search field: All Fields|. To check also articles dealing with with the topic, but not specifying it in the~title nor the abstract or keywords. This way, articles that aim more general - primarily to remote sensing or \zk{CNN}s - but dealing also with the second of the~subjects could be found.
	\item \verb|Publication years: 2020, 2019, 2018|. To focus only on recent publications.
	\item \verb|Web of Science categories: REMOTE SENSING|. To focus only on the~scien\-ti\-fic area of interest.
	\item \verb|Open Access: DOAJ Gold|. To focus only on articles and papers from sources listed on the~Di\-rectory of Open Access Journals\footnote{\url{www.doaj.org}} (\zk{DOAJ}).
\end{itemize}

\noindent The top five results from the query, when ordered by the~attribute \verb|Times Cited| and filtered as will be described below, were the following:

\begin{itemize}
	\item Building Extraction in Very High Resolution Remote Sensing Imagery Using Deep Learning and Guided Filters: 81 citations. \cite{res-u-net}
	\item Very Deep Convolutional Neural Networks for Complex Land Cover Mapping Using Multispectral Remote Sensing Imagery: 71 citations. \cite{very-deep-cnn-lc}
	\item Evaluation of Different Machine Learning Methods and Deep-Learning Convolutional Neural Networks for Landslide Detection: 51 citations. \cite{landslide-evaluation}
	\item 3D Convolutional Neural Networks for Crop Classification with Multi-Temporal Remote Sensing Images: 50 citations. \cite{3d-cnn-crop}
	\item Multi-Temporal Land Cover Classification with Sequential Recurrent Encoders: 38 citations. \cite{multi-temporal-sequential-recurrent}
\end{itemize}

\textit{Building Extraction in Very High Resolution Remote Sensing Imagery Using Deep Learning and Guided Filters} proposes a new \zk{CNN} architecture based on the residual network (\zk{ResNet}) \cite{resnet} called Res-U-Net and - using explicitly defined metrics - compares it to those of SegNet \cite{segnet}, fully convolutional network (\zk{FCN}) \cite{fcn}, a combination of \zk{CNN} and random forest (\zk{RF}) \cite{rf}, Multi-scale deep network \cite{hierarchical-labeling}, and a combination of \zk{CNN}, \zk{RF} and conditional random fields (\zk{CRF}) \cite{hierarchical-labeling}, but authors do not explain how exactly are these models built, so it does not give the~reader any solid comparison. Especially terms like \zk{CNN} and \zk{FCN} are so general that they do not imply anything more than just the basic approach. In the end, it is not even sure whether authors have tested these models on the~datasets on their own, or if they just report results found in other articles. Authors experiment on two datasets differing in used bands and spatial resolution, and also with including and excluding the normalized differential vegetation index (\zk{NDVI}) and the digital surface model (\zk{DSM}). However, both datasets are of German cities and therefore very similar in the~content, both also being binary datasets containg only two classes - \textit{building} and \textit{clutter} (unknown). Time and memory needs are not mentioned in the~study.

The second reviewed article - \textit{Very Deep Convolutional Neural Networks for Complex Land Cover Mapping Using Multispectral Remote Sensing Imagery} - deals with land cover mapping using \zk{CNN}s with a focus on wetlands detection. Therefore it is not a surprise that they do not experiment with different datasets; however, it would give much more general overview of \zk{CNN} possibilities in the wetland detection data from different parts of the world were used, and not only from Newfoundland and Labrador, Canada. Besides that, the paper experiments with DenseNet-121 \cite{densenet}, Inception V3 \cite{inception}, VGG-16 \cite{vgg}, VGG-19, Xception \cite{xception}, \zk{ResNet}-50, and Inception \zk{ResNet} V2 \cite{inception-resnet}, and also with machine learning (\zk{ML}) methods of support vector machine (\zk{SVM}) \cite{svm} and \zk{RF}. Authors also use different patch sizes and report processing times.

The third result contains the phrase \textit{evaluation of different methods} already in the title and compares \zk{CNN}s with other popular \zk{ML} methods, namely \zk{SVM}, \zk{RF}, and even with a simple artificial neural network (\zk{ANN}) architecture called multi-layer perceptron (\zk{MLP}) \cite{mlp} with a hidden layer of 30 neurons. Even the \zk{CNN} approach is diversified into two architectures - one of them comprising of five layers, the other one of seven layers. To make the results more general, authors experiment also with five different input window sizes, compare the~use of spectral bands versus the~use of a combination of spectral bands and topographical ones, and apply their models on two different datasets. Time consumption of different methods would be also valuable information on the comparison, yet this metric is not included in the article. It is apparent that authors of the paper came from the same impulse as this thesis, reading statements like \textit{"CNNs do not automatically outperform ANN, SVM and RF, although this is sometimes claimed. Rather, the performance of CNNs strongly depends on their design, i.e., layer depth, input window sizes and training strategies."} Or, in another place, \textit{"CNN will not automatically outperform other methods - as popular science articles and magazines may imply."} With felicitously set parameters, \zk{CNN}s outperformed other approaches, but their use was done under critical supervision and without trend-surfing shouts.

The fourth article, \textit{3D Convolutional Neural Networks for Crop Classification with Multi-Temporal Remote Sensing Images}, again experiments with different kernel sizes and other parameters, and compares the proposed 3D \zk{CNN} architecture with K-nearest neighbour (\zk{KNN}) \cite{knn}, principal component analysis (\zk{PCA}) \cite{pca}, \zk{SVM}, and an ad-hoc defined 2D \zk{CNN} structure. No time consumption analysis of different approaches was presented. Although the presented results seem to prove their claims that their proposed architecture performs better than the other ones, a more extensive research experimenting with more datasets and more advanced, state-of-the-art architectures could give such claims more solid position.

% The proposed method in this work can obtain improvements in terms of overall accuracy, precision and F_1 over the classical classification systems.

Authors of \textit{Multi-Temporal Land Cover Classification with Sequential Recurrent Encoders} do admit that their comparison part is not ideal; that is true as they compare results of their proposed architecture only with results of other architectures reported in other papers - these results were achieved using different datasets, different preprocessing, different numbers of classes and training samples, different spatial resolution and different bands. Therefore, it is arguable whether this comparison is relevant at all. Authors test their architecture on a multi-class dataset coming from one area in Germany split into two years; this also does not serve as a test on multiple independent datasets. Positive is the~fact that authors explicitly define used quality metrics and report time needs of the training, although only the~overall accuracy is presented for other architectures in the~comparison table.

Using the same query, but restrictive to search only in the topic (searching the title, the abstract, and keywords), the results were the same.

\subsubsection{Using the search term \textit{cnn}}
\label{wos-papers-cnn}

As in some articles, only the abbreviation \textit{\zk{CNN}} is used instead of the full-length term \textit{convolutional neural network}, the~same research using \textit{\zk{CNN}} as the search term was done.

\begin{itemize}
	\item \verb|Search string: cnn|. To focus only on articles dealing with \zk{CNN}s.
	\item \verb|Search field: All Fields|. To check also articles dealing with with the topic, but not specifying it in the~title nor the abstract or keywords. This way, articles that aim more general - primarily to remote sensing or \zk{CNN}s - but dealing also with the second of the~subjects could be found.
	\item \verb|Publication years: 2020, 2019, 2018|. To focus only on recent publications.
	\item \verb|Web of Science categories: REMOTE SENSING|. To focus only on the~scien\-ti\-fic area of interest.
	\item \verb|Open Access: DOAJ Gold|. To focus only on articles and papers from sources listed on the~Di\-rectory of Open Access Journals\footnote{\url{www.doaj.org}} (\zk{DOAJ}).
\end{itemize}

\noindent The top five results from the query, when ordered by the~attribute \verb|Times Cited| and filtered as will be described below, were the following:

\begin{itemize}
	\item Evaluation of Different Machine Learning Methods and Deep-Learning Convolutional Neural Networks for Landslide Detection: 51 citations. \cite{landslide-evaluation}
	\item 3D Convolutional Neural Networks for Crop Classification with Multi-Temporal Remote Sensing Images: 50 citations. \cite{3d-cnn-crop}
	\item Geospatial Object Detection in High Resolution Satellite Images Based on Multi-Scale Convolutional Neural Network: 32 citations. \cite{object-detection-hrs-multi-scale}
	\item Hyperspectral Image Classification Using Convolutional Neural Networks and Multiple Feature Learning: 31 citations. \cite{hyperspectral-multiple-feat-cnn}
	\item Deformable Faster R-CNN with Aggregating Multi-Layer Features for Partially Occluded Object Detection in Optical Remote Sensing Images: 24 citations. \cite{deformable-faster-r-cnn}
\end{itemize}

Two results were filtered out. \cite{cnn-fusion-clouds} with 28 citations and \cite{cnn-fusion-hr-hsr} with 25 citations. They were filtered out only because they do not correspond with the~main focus of the~thesis on the classification task with object detection and semantic and instance segmentation, but dealt with an image fusion instead. The top two results correspond to articles reviewed already in the previous section; therefore they are being skipped now.

The comparison part of \textit{Geospatial Object Detection in High Resolution Satellite Images Based on Multi-Scale Convolutional Neural Network} is a bit more minimalistic. The proposed method is compared only with an architecture called Faster \zk{R-CNN} (region based convolutional neural network) \cite{faster-rcnn} and with the single shot multi-box detector (\zk{SSD}) \cite{ssd} without a more detailed description of inner parametrization or experiments with it. Authors use also only one dataset to test their method, so there is no evidence on how versatile the method is when it comes to the data greed. On the other hand, they report the time greed of the three used methods.

In \textit{Hyperspectral Image Classification Using Convolutional Neural Networks and Multiple Feature Learning}, authors go a~different way - to compare their architecture, they create two different architectures and show that the~proposed one is the~one performing the best. A comparison with any other well-known architecture or other \zk{ML} method is missing. The pro of this paper is the use of three different datasets.

24 citations reaching \textit{Deformable Faster R-CNN with Aggregating Multi-Layer Features for Partially Occluded Object Detection in Optical Remote Sensing Images} also proposes a new architecture, called \textit{deformable R-CNN}. Authors compare this model with \zk{SSD} and R-P-Faster \zk{R-CNN} \cite{rp-faster-rcnn} on three datasets. No comparison of time or memory requirements is included in the article.

Using the same query, but restrictive to search only in the topic (searching the title, the abstract, and keywords), the results were the same.

% Convolution Neural Network Architecture Learning for Remote Sensing Scene Classification
% https://arxiv.org/abs/2001.09614

% Murthy, C.; Raju, P.; Badrinath, K. Classification of wheat crop with multi-temporal images: Performance of maximum likelihood and artificial neural networks. Int. J. Remote Sens. 2003, 24, 4871–4890. ?????????????????????????????

% Lin, H.; Shi, Z.; Zou, Z. Maritime Semantic Labeling of Optical Remote Sensing Images with Multi-Scale Fully Convolutional Network. ????????????

% WoS #2
% Geospatial Object Detection in High Resolution Satellite Images Based on Multi-Scale Convolutional Neural Network
% https://apps.webofknowledge.com/full_record.do?product=WOS&search_mode=GeneralSearch&qid=8&SID=C6dRsuZprCvu65HT7SF&page=1&doc=2&cacheurlFromRightClick=no
% superiority of the proposed method

% #3
% superior performances of the proposed framework

% #4
% Deformable Faster R-CNN with Aggregating Multi-Layer Features for Partially Occluded Object Detection in Optical Remote Sensing Images
% https://apps.webofknowledge.com/full_record.do?product=WOS&search_mode=GeneralSearch&qid=5&SID=C6pjuytIsVIIBFmwnOt&page=1&doc=4&cacheurlFromRightClick=no
% compared with SSD, R-P-Faster R-CNN, also compared on three datasets (NWPU, SORSI, HRRS), no comparison on time or memory usage

% #5
% Ship Detection Based on YOLOv2 for SAR Imagery
% https://apps.webofknowledge.com/full_record.do?product=WOS&search_mode=GeneralSearch&qid=5&SID=C6pjuytIsVIIBFmwnOt&page=1&doc=5&cacheurlFromRightClick=no
% compared with Faster R-CNN and not with YOLOv2 or YOLO, two datasets (both about ships), compared time needs
% "YOLOv2-reduced is best for real time object detection problem."

\subsection{Scopus}
\label{scopus-papers}

\subsubsection{Using the search term \textit{convolutional neural network}}
\label{scopus-papers-full-length}

The second used website was Scopus\footnote{\url{www.scopus.com}}. The~following query restrictions were used:

\begin{itemize}
	\item \verb|Search string: "remote sensing" "convolutional neural network"|. To focus only on articles dealing with \zk{CNN}s in the area of remote sensing.
	\item \verb|Search field: All fields|. To check also articles dealing with with the topic, but not specifying it in the~title nor the abstract or keywords. This way, articles that aim more general - primarily to remote sensing or \zk{CNN}s - but dealing also with the second of the~subjects could be found.
	\item \verb|Publication years: 2020, 2019, 2018|. To focus only on recent publications.
	\item \verb|Access type: Open Access|. To focus only on articles in \textit{Scopus Gold Open Access}\footnote{\url{www.elsevier.com/open-access}}. It includes fully open journals, hybrid journals (authors pay a fee to make an article open access), open archives and articles with free promotional access.
\end{itemize}

\noindent The Scopus query is apparently - probably due to the~smaller flexibility when it comes to the~open access restrictions - more tolerant and includes articles filtered out from the~\zk{WoS} query. Also, because there is no scientific category \verb|remote sensing| in the~Scopus search engine, the~phrase was included in the~search phrase and more manual filtering was needed, as will be described below. The~top five results from the query, when ordered by the~attribute \verb|Cited by|, were the following:

\begin{itemize}
	\item A new deep convolutional neural network for fast hyperspectral image classification: 123 citations.  \cite{cnn-hs-class}
	\item Automatic ship detection in remote sensing images from google earth of complex scenes based on multiscale rotation Dense Feature Pyramid Networks: 97 citations. \cite{ship-rdfpn}
	\item Deep learning in remote sensing applications: A meta-analysis and review: 93 citations. \cite{dl-remote-sensing-review}
	\item Building extraction in very high resolution remote sensing imagery using deep learning and guided filters: 92 citations. \cite{res-u-net}
	\item Semi-Supervised Deep Learning Using Pseudo Labels for Hyperspectral Image Classification: 77 citations \cite{semi-supervised-hyperspectral}
\end{itemize}

Ten results were filtered out. \cite{dl-for-cv} with 251 citations, \cite{ir-image-fusion} with 200 citations, \cite{review-ml-in-rs} with 177 citations, \cite{rf-knn-svm-for-lc} with 157 citations, \cite{review-ml-agriculture} with 153 citations, \cite{review-uav-applications} with 104 citations, \cite{review-survey-dl} with 98 citations, \cite{dl-medicine-preprocessing} with 87 citations, \cite{state-of-the-art-ann} with 85 citations, and \cite{lc-2-0} with 83 citations. They were filtered out only because they do not correspond with the~main focus of the~thesis on the classification task with object detection and semantic and instance segmentation in the field of remote sensing.

The first article, \textit{A new deep convolutional neural network for fast hyperspectral image classification}, starts the~research again in a very positive way. Authors have compared their own architecture with the \zk{MLP} and three different \zk{CNN}s - a one-dimensional one, a two-dimensional one and a three-dimensional one. Although a~comparison with some popular architectures that just haven't been used on hyperspectral images or some classical \zk{ML} methods would be also interesting, the used ones are properly sourced to another research on hyperspectral image classification and authors underlined main differences between the proposed model and the ones used for evaluation. All experiments have been conducted on two datasets differing in the number of bands, pixel size of images, spatial resolution, and also in objects of classification. Authors also experiment with different patch sizes and - importantly - with the~number of samples per class.

\textit{Automatic ship detection in remote sensing images from google earth of complex scenes based on multiscale rotation Dense Feature Pyramid Networks} also does not compare the~proposed methodology with basic \zk{ML} approaches, but uses a lot of \zk{CNN} models to compare their architecture with - Faster \zk{R-CNN}, Feature pyramid network (\zk{FPN}) \cite{fpn}, rotation region proposal network (\zk{RRPN}) \cite{rrpn}, and rotational region convolutional neural network (\zk{R2CNN}) \cite{r2cnn}. However, as the~paper focuses only on the~ship detection, there is no experiment on other datasets.

The third most cited article on Scopus - \textit{Deep learning in remote sensing applications: A meta-analysis and review} - is a review coming from similar impulses as this thesis or partly \cite{landslide-evaluation} mentioned in chapter \ref{wos-papers}. The motivation is formulated in the sense that \textit{"it appears that a more systematic (i.e. quantitative) analysis is necessary to get a comprehensive and objective understanding of the applications of DL for remote-sensing analysis."}. Authors focus on many more fields than is the goal of this thesis, including image fusion, image registration, scene classification, object detection, land use and land cover classification, image segmentation, object-based image analysis (\zk{OBIA}), and other tasks. The~value of the~paper lies in its extensive research on what are the~most frequent targets of the~use of deep learning (\zk{DL}) for remote sensing, what are the~most frequent \zk{DL} models, spatial resolutions, application areas (urban, vegetation, etc.), average accuracies, common training datasets and even the~most used scientific terms in titles and abstracts of these papers. Although it is a high-class review and it works as a valuable overview about \zk{DL} stronger and weaker positions in the~field, authors have not conducted their own experiments, so only results reported in the~original papers are mentioned, if mentioned.

\textit{Building extraction in very high resolution remote sensing imagery using deep learning and guided filters} is the same one as the one mentioned in chapter \ref{wos-papers}, only with different number of citations due to the difference between the \zk{WoS} and Scopus systems; therefore it is being skipped in this section.

Authors of \textit{Semi-Supervised Deep Learning Using Pseudo Labels for Hyperspectral Image Classification} propose their own method called semi-supervised deep learning using pseudo-labels (\zk{PL-SSDL}) using convolutional recurrent neural networks (\zk{CRNN}) and compare its performace on three multi-class hyperspectral datasets differing in number of classes, number of bands, and also in the~spatial resolution (ranging from 1 metre to 2.5 metres). \zk{PL-SSDL} is compared with big amount of \zk{ML} methods, namely \zk{KNN}, \zk{SVM}, label propagation \cite{label-prop}, transductive support vector machine (\zk{TSVM}) \cite{tsvm}, Laplacian support vector machine (\zk{LapSVM}) \cite{lapsvm}, and spatio-spectral Laplacian support vector machine (\zk{SS-LapSVM}) \cite{ss-lapsvm}, and two \zk{ANN} architectures, namely stacked denoising autoencoder (\zk{SDA}) \cite{sda}, and ladder networks \cite{ladder-networks}. As \zk{PL-SSDL} is a \zk{CRNN}, a comparison with \zk{CNN} or \zk{RNN} or their combination would be desirable, especially considering the fact that the used \zk{ANN} models could be built using them. Authors report for every method time complexity and accuracy , but do not specify explicitly what accuracy metric do they use or how do they measure the time complexity (number of epochs is slightly mentioned for \zk{PL-SSDL}, but not for any other approach).

\subsubsection{Using the search term \textit{cnn}}
\label{scopus-papers-cnn}

As in some articles, only the abbreviation \textit{\zk{CNN}} is used instead of the full-length term \textit{convolutional neural network}, the~same research using \textit{\zk{CNN}} as the search term was done.

\begin{itemize}
	\item \verb|Search string: "remote sensing" cnn|. To focus only on articles dealing with \zk{CNN}s in the area of remote sensing.
	\item \verb|Search field: All fields|. To check also articles dealing with with the topic, but not specifying it in the~title nor the abstract or keywords. This way, articles that aim more general - primarily to remote sensing or \zk{CNN}s - but dealing also with the second of the~subjects could be found.
	\item \verb|Publication years: 2020, 2019, 2018|. To focus only on recent publications.
	\item \verb|Access type: Open Access|. To focus only on articles in \textit{Scopus Gold Open Access}\footnote{\url{www.elsevier.com/open-access}}. It includes fully open journals, hybrid journals (authors pay a fee to make an article open access), open archives and articles with free promotional access.
\end{itemize}

\noindent The Scopus query is apparently - due to the~smaller flexibility when it comes to the~open access restrictions - more tolerant and includes articles filtered out from the~\zk{WoS} query. Also, because there is no scientific category \verb|remote sensing| in the~Scopus search engine, the~phrase was included in the~search phrase and more manual filtering was needed, as will be described below. The~top five results from the query, when ordered by the~attribute \verb|Cited by|, were the following:

\begin{itemize}
	\item A new deep convolutional neural network for fast hyperspectral image classification: 123 citations.  \cite{cnn-hs-class}
	\item Automatic ship detection in remote sensing images from google earth of complex scenes based on multiscale rotation Dense Feature Pyramid Networks: 97 citations. \cite{ship-rdfpn}
	\item Deep learning in remote sensing applications: A meta-analysis and review: 93 citations. \cite{dl-remote-sensing-review}
	\item Building extraction in very high resolution remote sensing imagery using deep learning and guided filters: 92 citations. \cite{res-u-net}
	\item Very Deep Convolutional Neural Networks for Complex Land Cover Mapping Using Multispectral Remote Sensing Imagery: 72 citations. \cite{very-deep-cnn-lc}
\end{itemize}

Four results were filtered out. \cite{dl-for-cv} with 251 citations, \cite{dl-lungs} with 87 citations, \cite{maoxian-landslide} with 74 citations, and \cite{state-of-the-art-dl} with 73 citations. They were filtered out only because they do not correspond with the~main focus of the~thesis on the classification task with object detection and semantic and instance segmentation in the field of remote sensing.

From the~results above, the top four results correspond to articles reviewed already in the previous section and the fifth one to an article reviewed already in chapter \ref{wos-papers}, only with different number of citations due to the difference between the \zk{WoS} and Scopus systems. Therefore, all of them are being skipped in this section.

\section{Reviews of CNNs in the field of remote sensing}
\label{situation}

\subsection{Web of Science}
\label{wos-reviews}

\subsubsection{Using the search term \textit{convolutional neural network}}
\label{wos-reviews-full-length}

When compared with the query described in chapter \ref{wos-papers}, an extra parameter \verb|Document Types| was used to get only \textit{review-type} articles. Therefore, the following restrictions were used for the~query:

\begin{itemize}
	\item \verb|Search string: convolutional neural network|. To focus only on articles dealing with \zk{CNN}s.
	\item \verb|Search field: All Fields|. To check also articles dealing with with the topic, but not specifying it in the~title nor the abstract or keywords. This way, articles that aim more general - primarily to remote sensing or \zk{CNN}s - but dealing also with the second of the~subjects could be found.
	\item \verb|Publication years: 2020, 2019, 2018|. To focus only on recent publications.
	\item \verb|Web of Science categories: REMOTE SENSING|. To focus only on the~scien\-ti\-fic area of interest.
	\item \verb|Open Access: DOAJ Gold|. To focus only on articles and papers from sources listed on the~Di\-rectory of Open Access Journals\footnote{\url{www.doaj.org}} (\zk{DOAJ}).
	\item \verb|Document Types: REVIEW|. To focus only on reviews.
\end{itemize}

\noindent There were only seven results for the~described query and two of them - \textit{Spatiotemporal Image Fusion in Remote Sensing} \cite{review-st-fusion} with 14 citations and \textit{Meta-Analysis of Wetland Classification Using Remote Sensing: A Systematic Review of a 40-Year Trend in North America} \cite{review-wetlands-40-years} with 0 citations - were filtered out as they dealt with the~image fusion and with the history of wetland classification and did not overlay with the main focus of this thesis. The rest, when ordered by the~attribute \verb|Times Cited|, were the following:

\begin{itemize}
	\item Review and Evaluation of Deep Learning Architectures for Efficient Land Cover Mapping with UAS Hyper-Spatial Imagery: A Case Study Over a Wetland: 3 citations. \cite{review-dl-wetlands}
	\item Object Detection and Image Segmentation with Deep Learning on Earth Observation Data: A Review-Part I: Evolution and Recent Trends: 1 citation. \cite{review-dl-eo}
	\item Geographic Object-Based Image Analysis: A Primer and Future Directions: 0 citations. \cite{geobia}
	\item Deep Learning for Land Use and Land Cover Classification Based on Hyperspectral and Multispectral Earth Observation Data: A Review: 0 citations. \cite{review-dl-lulc}
	\item Deep Learning Approaches Applied to Remote Sensing Datasets for Road Extraction: A State-Of-The-Art Review: 0 citations. \cite{review-dl-road-extraction}
\end{itemize}

As \cite{very-deep-cnn-lc}, \textit{Review and Evaluation of Deep Learning Architectures for Efficient Land Cover Mapping with UAS Hyper-Spatial Imagery: A Case Study Over a Wetland} focuses on wetlands. And although authors evaluate and describe a pleasurable amount of architectures including SegNet, U-Net \cite{u-net}, DenseNet, DeepLab V3+ \cite{deeplab}, Pyramid scene parsing network (\zk{PSPNet}) \cite{pspnet}, and an architecture called MobileU-Net (a combination of MobileNet \cite{mobilenet} and U-Net) and dedicate a lot of space to used evaluation metrics, their findings would be even more valuable and universal if tested on more than one dataset. An application of at least one of the~common \zk{ML} methods would also give the~article an added value in the form of a~well-understood relative comparison.

\textit{Object Detection and Image Segmentation with Deep Learning on Earth Observation Data: A Review-Part I} (part two is as of August 14, 2020 not yet published) focuses on an overview and the~evolution of different approaches in the topic. Authors present a comprehensive overview of \zk{DL} architectures commonly or less commonly used on Earth observation (\zk{EO}) data, but - due to their main goal being just an overview - do not conduct their own experiments; instead, they report results based on the~\zk{MS-COCO} dataset (Microsoft-Common Objects in Context) \cite{coco}, a dataset with contents very different than those of \zk{EO} data.

The main focus of \textit{Geographic Object-Based Image Analysis: A Primer and Future Directions} is on the geographic object-based analysis (\zk{GEOBIA}) and not \zk{CNN}s; however, a not negligible part of the~review deals with \zk{CNN}s and \zk{CNN}s - called geographic object-based convolutional neural networks (\zk{GEOCNN}s) - are proposed as one of the~most promising future directions of \zk{GEOBIA}. But when it comes to the~chapter \textit{Accuracies of GEOCNN Methods Versus Conventional GEOBIA}, no experiments are conducted and the~entire report is reduced to citations with trifling claims without numbers like \textit{"method following Approach 3 resulted in higher thematic accuracy than per-pixel classification with patch-based CNNs and FCNs"} or \textit{"method following Approach 2 resulted in higher segmentation accuracy than GEOBIA"}.

As in \cite{review-dl-eo}, authors of \textit{Deep Learning for Land Use and Land Cover Classification Based on Hyperspectral and Multispectral Earth Observation Data: A Review} also do not conduct any experiment, but rather report the evolution of the \zk{DL} and land cover and land use classification. And although they sketch the connection between different topic problems and different \zk{ML} or \zk{DL} architectures, they do not report any numbers and when they rarely write about about performances, only general phrases like \textit{"U-Net has shown very promising results in extracting buildings"} are used.

\textit{Deep Learning Approaches Applied to Remote Sensing Datasets for Road Extraction: A State-Of-The-Art Review} introduces different approaches to road extraction using \zk{DL}. Although it gives always multiple model examples for every approach, their results - crucial for the right model choice - are taken from original papers; therefore, different metrics are used, time needs are mentioned only exceptionally, and different datasets are used, and even when the same dataset is used, different setting (as the training-validation images ratio) could lead to different scores.

\subsubsection{Using the search term \textit{cnn}}
\label{wos-reviews-cnn}

As in some articles, only the abbreviation \textit{\zk{CNN}} is used instead of the full-length term \textit{convolutional neural network}, the~same research using \textit{\zk{CNN}} as the search term was done.

\begin{itemize}
	\item \verb|Search string: cnn|. To focus only on articles dealing with \zk{CNN}s. The~term \verb|cnn| was preferred over the~term \verb|convolutional neural network| as in some papers, only the~abbreviated form was used (this does not apply to papers only marginally mentioning \zk{CNN}s, where the term is normally full-length).
	\item \verb|Search field: All Fields|. To check also articles dealing with with the topic, but not specifying it in the~title nor the abstract or keywords. This way, articles that aim more general - primarily to remote sensing or \zk{CNN}s - but dealing also with the second of the~subjects could be found.
	\item \verb|Publication years: 2020, 2019, 2018|. To focus only on recent publications.
	\item \verb|Web of Science categories: REMOTE SENSING|. To focus only on the~scien\-ti\-fic area of interest.
	\item \verb|Open Access: DOAJ Gold|. To focus only on articles and papers from sources listed on the~Di\-rectory of Open Access Journals\footnote{\url{www.doaj.org}} (\zk{DOAJ}).
	\item \verb|Document Types: REVIEW|. To focus only on reviews.
\end{itemize}

\noindent There were only five results for the~described query and one of them - \textit{Spatiotemporal Image Fusion in Remote Sensing} \cite{review-st-fusion} with 14 citations - was filtered out as it dealt with the~image fusion and not with the~classification. The rest, when ordered by the~attribute \verb|Times Cited|, were the following:

\begin{itemize}
	\item Review and Evaluation of Deep Learning Architectures for Efficient Land Cover Mapping with UAS Hyper-Spatial Imagery: A Case Study Over a Wetland: 3 citations. \cite{review-dl-wetlands}
	\item UAV-Based Structural Damage Mapping: A Review: 3 citations. \cite{uav-building-damages}
	\item Object Detection and Image Segmentation with Deep Learning on Earth Observation Data: A Review-Part I: Evolution and Recent Trends: 1 citation. \cite{review-dl-eo}
	\item Geographic Object-Based Image Analysis: A Primer and Future Directions: 0 citation. \cite{geobia}
\end{itemize}

The only review not described in the previous section is the second one. \textit{UAV-Based Structural Damage Mapping: A Review} does not focus primarily on \zk{CNN}s, but as they constitute a big part of UAV-based damage mapping, there is a lot of reviewing of them in the~article. However, the~article works more as an overview of the evolution of different approaches in the topic. Therefore, no novel experiments were conducted and when some results are mentioned, they are only copied from original papers; papers, where these methods could have been used on different data with different results.

\subsection{Scopus}
\label{scopus-reviews}

\subsubsection{Using the search term \textit{convolutional neural network}}
\label{scopus-reviews-full-length}

When compared with the query described in chapter \ref{scopus-papers}, an extra parameter \verb|Document type| was used to get only \textit{review-type} articles. Therefore, the following restrictions were used for the~query:

\begin{itemize}
	\item \verb|Search string: "remote sensing" "convolutional neural network"|. To focus only on articles dealing with \zk{CNN}s in the area of remote sensing.
	\item \verb|Search field: All fields|. To check also articles dealing with with the topic, but not specifying it in the~title nor the abstract or keywords. This way, articles that aim more general - primarily to remote sensing or \zk{CNN}s - but dealing also with the second of the~subjects could be found.
	\item \verb|Publication years: 2020, 2019, 2018|. To focus only on recent publications.
	\item \verb|Access type: Open Access|. To focus only on articles in \textit{Scopus Gold Open Access}\footnote{\url{www.elsevier.com/open-access}}. It includes fully open journals, hybrid journals (authors pay a fee to make an article open access), open archives and articles with free promotional access.
	\item \verb|Document type: Review|. To focus only on reviews.
\end{itemize}

The Scopus query is apparently - due to the~smaller flexibility when it comes to the~open access restrictions - more tolerant and includes articles filtered out from the~\zk{WoS} query. Also, because there is no scientific category \verb|remote sensing| in the~Scopus search engine, the~phrase was included in the~search phrase. It resulted in the~fact that from the~top 20 articles when orded by the~attribute \verb|Cited by|, 19 were filtered out as they either did not correspond with the~main focus of the~thesis on the classification task with object detection and semantic and instance segmentation in the field of remote sensing, or they were found completely insufficient in terms of providing a~comparison of mentioned architectures.

The 19 filtered articles were the following: \cite{dl-for-cv} with 251 citations, \cite{review-ml-in-rs} with 177 citations, \cite{review-ml-agriculture} with 153 citations, \cite{review-uav-applications} with 104 citations, \cite{state-of-the-art-ann} with 85 citations, \cite{lc-2-0} with 83 citations, \cite{state-of-the-art-dl} with 73 citations, \cite{review-water-dl} with 71 citations, \cite{review-plant-stress} with 66 citations, \cite{review-text-class} with 53 citations, \cite{cv-animal-ecology} with 50 citations, \cite{review-cv-infra-inspections} with 49 citations, \cite{review-ann-plant-disease} with 48 citations, \cite{review-st-fusion-multisource} with 48 citations, \cite{review-ml-energy} with 47 citations, \cite{review-bd-disaster} with 43 citations, \cite{review-uav-rs} with 34 citations, \cite{nanophotonics} with 33 citations, and \cite{review-point-clouds} with 33 citations.

The only on corresponding with the aim of this thesis is \textit{Deep learning for remote sensing image classification: A survey} \cite{review-dl-rs} with 45 citations. The approach is very fulfilling - authors compare multiple architectures on multiple datasets and define metrics used for the comparison. The fact that almost half of the presented results are taken from a different source is not a problem, but the fact that in the other part, a different training-validation-test data ratio was used for different models is. Also, the researched architectures are just vaguely defined without specifying parameters used to build them.

\subsubsection{Using the search term \textit{cnn}}
\label{scopus-reviews-cnn}

As in some articles, only the abbreviation \textit{\zk{CNN}} is used instead of the full-length term \textit{convolutional neural network}, the~same research using \textit{\zk{CNN}} as the search term was done.

\begin{itemize}
	\item \verb|Search string: "remote sensing" cnn|. To focus only on articles dealing with \zk{CNN}s in the area of remote sensing.
	\item \verb|Search field: All fields|. To check also articles dealing with with the topic, but not specifying it in the~title nor the abstract or keywords. This way, articles that aim more general - primarily to remote sensing or \zk{CNN}s - but dealing also with the second of the~subjects could be found.
	\item \verb|Publication years: 2020, 2019, 2018|. To focus only on recent publications.
	\item \verb|Access type: Open Access|. To focus only on articles in \textit{Scopus Gold Open Access}\footnote{\url{www.elsevier.com/open-access}}. It includes fully open journals, hybrid journals (authors pay a fee to make an article open access), open archives and articles with free promotional access.
	\item \verb|Document type: Review|. To focus only on reviews.
\end{itemize}

The Scopus query is apparently - due to the~smaller flexibility when it comes to the~open access restrictions - more tolerant and includes articles filtered out from the~\zk{WoS} query. Also, because there is no scientific category \verb|remote sensing| in the~Scopus search engine, the~phrase was included in the~search phrase. It resulted in the~fact that from the~top 20 articles when orded by the~attribute \verb|Cited by|, 19 were filtered out as they either did not correspond with the~main focus of the~thesis on the classification task with object detection and semantic and instance segmentation in the field of remote sensing, or they were found completely insufficient in terms of providing a~comparison of mentioned architectures.

The 19 filtered articles were the following: \cite{dl-for-cv} with 251 citations, \cite{state-of-the-art-dl} with 73 citations, \cite{review-water-dl} with 71 citations, \cite{review-plant-stress} with 66 citations, \cite{review-text-class} with 53 citations, \cite{review-oil-spill} with 50 citations, \cite{review-cv-infra-inspections} with 49 citations, \cite{review-vessel-detection} with 39 citations, \cite{review-ml-smart-grid} with 30 citations, \cite{review-autonomus-vehicles} with 28 citations, \cite{review-grasp} with 21 citations, \cite{review-3d-human-interaction} with 21 citations, \cite{review-dl-3d-classification} with 19 citations, \cite{review-crop-phenomics} with 19 citations, \cite{review-age-estimation} with 18 citations, \cite{review-shot-boundary} with 17 citations, \cite{review-crop-phenomics-breeding} with 16 citations, \cite{review-video-crowd} with 15 citations, and \cite{review-dl-food} with 15 citations.

The only article that had a bigger than an insignificant overlap with the goal of this thesis was the ninth result, \textit{Deep learning meets hyperspectral image analysis: A multidisciplinary review} \cite{review-dl-hs-rs-bio}. Tha aim of this review is to give a~general overview of the~use of \zk{DL} in fields of remote sensing and biomedicine and it does so by naming and referencing an impressive amount of papers, but without any abundant architecture list and without original experiments. Not even citations of results from original papers could be found.
