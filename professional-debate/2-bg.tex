\chapter{Motivation}
\label{motivation}

In August 2020, an experiment was conducted - a query for papers with the below-described restrictions was done on selected academic database websites. The goal of the experiment was to research whether authors of papers dealing with the use of convolutional neural networks (\zk{CNN}) in the field of remote sensing are comparing results of proposed or used architectures with results of other architectures on the same task.

\section{Web of Science}
\label{wos}

The first used website was Web of Science\footnote{\url{www.webofknowledge.com}} (\zk{WoS}). The~following query restrictions were used:

\begin{itemize}
	\item \verb|Search string: cnn|. To focus only on articles dealing with \zk{CNN}s. The~term \verb|cnn| was preferred over the~term \verb|convolutional neural network| as in some papers, only the~abbreviated form was used.
	\item \verb|Publication years: 2020, 2019, 2018|. To focus only on recent publications.
	\item \verb|Web of Science categories: REMOTE SENSING|. To focus only on the~scientific area of interest.
	\item \verb|Open Access: DOAJ Gold|. To focus only on articles from journals listed on the~Directory of Open Access Journals\footnote{\url{www.doaj.org}} (\zk{DOAJ}). 
\end{itemize}

\noindent The first five results from the query, when ordered by the~attribute \verb|Times Cited| and filtered as will be described below, were the following:

\begin{itemize}
	\item Evaluation of Different Machine Learning Methods and Deep-Learning Convolutional Neural Networks for Landslide Detection: 51 citations.  \cite{landslide-evaluation}
	\item 3D Convolutional Neural Networks for Crop Classification with Multi-Temporal Remote Sensing Images: 50 citations. \cite{3d-cnn-crop}
	\item Geospatial Object Detection in High Resolution Satellite Images Based on Multi-Scale Convolutional Neural Network: 32 citations. \cite{object-detection-hrs-multi-scale}
	\item Hyperspectral Image Classification Using Convolutional Neural Networks and Multiple Feature Learning: 31 citations. \cite{hyperspectral-multiple-feat-cnn}
	\item Deformable Faster R-CNN with Aggregating Multi-Layer Features for Partially Occluded Object Detection in Optical Remote Sensing Images: 24 citations. \cite{deformable-faster-r-cnn}
\end{itemize}

Two results were filtered out. \cite{cnn-fusion-clouds} with 28 citations and \cite{cnn-fusion-hr-hsr} with 25 citations. They were filtered out only because they do not correspond with the~main focus of the~thesis on the classification task with object detection and semantic and instance segmentation, but dealt with an image fusion instead. 

The research starts very positively. The first result contains the phrase \textit{evaluation of different methods} already in the title and compares \zk{CNN}s with other popular machine learning (\zk{ML}) methods, namely support vector machine (\zk{SVM}) \cite{svm}, random forest (\zk{RF}) \cite{rf} and even with a simple artificial neural network (\zk{ANN}) architecture called multi-layer perceptron (\zk{MLP}) \cite{mlp} with a hidden layer of 30 neurons. Even the \zk{CNN} approach is diversified into two architectures - one of them comprising of five layers, the other one of seven layers. To make the results even more general, authors experiments also with five different input window sizes, compare the~use of spectral bands versus the~use of a combination of spectral bands and topograhical ones and apply their models on two different datasets. Time consumption of different methods would be also valuable information on the comparison, yet this one is metric is not included in the article. It is apparent that authors of the paper came from the same impulse as this thesis, reading statements like \textit{"CNNs do not automatically outperform ANN, SVM and RF, although this is sometimes claimed. Rather, the performance of CNNs strongly depends on their design, i.e., layer depth, input window sizes and training strategies."} Or, in another place, \textit{"CNN will not automatically outperform other methods - as popular science articles and magazines may imply."} With felicitously set parameters, \zk{CNN}s outperformed other approaches, but their use was done under a critical supervision and without trend-surfing shouts.

The second article, \textit{3D Convolutional Neural Networks for Crop Classification with Multi-Temporal Remote Sensing Images},  again experiments with different kernel sizes and other parameters, and compares the proposed 3D \zk{CNN} architecture with K-nearest neighbour (\zk{KNN}) \cite{knn}, principal component analysis (\zk{PCA}) \cite{pca}, \zk{SVM}, and an ad-hoc defined 2D \zk{CNN} structure. No time consumption analysis of different approaches was presented. Although the presented results seem to proof their claims that their proposed architecture performs better than the other ones, a more extensive research experimenting with more datasets and more advanced, state-of-the-art architectures could give such claims more solid position. 

The comparison part of \textit{Geospatial Object Detection in High Resolution Satellite Images Based on Multi-Scale Convolutional Neural Network} is a bit more minimalistic. The proposed method is compared only with an architecture called Faster \zk{R-CNN} (region based convolutional neural network) \cite{faster-rcnn} and with the single shot multi-box detector (\zk{SSD}) \cite{ssd} without more detailed description of inner parametrization or experiments with it. Authors use also only one dataset to test their method, so there is no evidence on how versatile the method is when it comes to the data greed. On the other hand, they report the time greed of the three used methods. 

In \textit{Hyperspectral Image Classification Using Convolutional Neural Networks and Multiple Feature Learning}, authors go different way - to compare their architecture, they make up two different architectures and show that the one proposed is the one performing the best. A comparison with any other well-known architecture or other \zk{ML} method is missing. The pro of this paper is the use of three different datasets.

24 citations reaching \textit{Deformable Faster R-CNN with Aggregating Multi-Layer Features for Partially Occluded Object Detection in Optical Remote Sensing Images} also proposes a new architecture, called \textit{deformable R-CNN}. Authors compare this model with \zk{SSD} and R-P-Faster \zk{R-CNN} \cite{rp-faster-rcnn} on three datasets. No comparison of time or memory requirements is included in the article.


% Convolution Neural Network Architecture Learning for Remote Sensing Scene Classification
% https://arxiv.org/abs/2001.09614

% Murthy, C.; Raju, P.; Badrinath, K. Classification of wheat crop with multi-temporal images: Performance of maximum likelihood and artificial neural networks. Int. J. Remote Sens. 2003, 24, 4871–4890. ?????????????????????????????

% Lin, H.; Shi, Z.; Zou, Z. Maritime Semantic Labeling of Optical Remote Sensing Images with Multi-Scale Fully Convolutional Network. ????????????

% WoS #2
% Geospatial Object Detection in High Resolution Satellite Images Based on Multi-Scale Convolutional Neural Network
% https://apps.webofknowledge.com/full_record.do?product=WOS&search_mode=GeneralSearch&qid=8&SID=C6dRsuZprCvu65HT7SF&page=1&doc=2&cacheurlFromRightClick=no
% superiority of the proposed method

% #3
% superior performances of the proposed framework

% #4
% Deformable Faster R-CNN with Aggregating Multi-Layer Features for Partially Occluded Object Detection in Optical Remote Sensing Images
% https://apps.webofknowledge.com/full_record.do?product=WOS&search_mode=GeneralSearch&qid=5&SID=C6pjuytIsVIIBFmwnOt&page=1&doc=4&cacheurlFromRightClick=no
% compared with SSD, R-P-Faster R-CNN, also compared on three datasets (NWPU, SORSI, HRRS), no comparison on time or memory usage

% #5
% Ship Detection Based on YOLOv2 for SAR Imagery
% https://apps.webofknowledge.com/full_record.do?product=WOS&search_mode=GeneralSearch&qid=5&SID=C6pjuytIsVIIBFmwnOt&page=1&doc=5&cacheurlFromRightClick=no
% compared with Faster R-CNN and not with YOLOv2 or YOLO, two datasets (both about ships), compared time needs
% "YOLOv2-reduced is best for real time object detection problem."

\section{Scopus}
\label{scopus}

The second used website was Scopus\footnote{\url{www.scopus.com}}. The~following query restrictions were used:

\begin{itemize}
	\item \verb|Search string: "remote sensing" cnn|. To focus only on articles dealing with \zk{CNN}s. The~term \verb|cnn| was preferred over the~term \verb|convolutional neural network| as in some papers, only the~abbreviated form was used.
	\item \verb|Publication years: 2020, 2019, 2018|. To focus only on recent publications.
	\item \verb|Acces type: Open Access|. To focus only on articles in \textit{Scopus Gold Open Access}\footnote{\url{www.elsevier.com/open-access}}. It includes fully open journals, hybrid journals (authors pay a fee to make an article open access), open archives and artiles with free promotional access.
\end{itemize}

\noindent The Scopus query is apparently - due to the~smaller flexibility when it comes to the~open access restrictions - more tolerant and includes articles filtered out from the~\zk{WoS} query. Also, because there is no scientific category \verb|remote sensing| in the~Scopus search engine, the~phrase was included in the~search phrase and more manual filtering was needed, as will be described below. The~first five results from the query, when ordered by the~attribute \verb|Times Cited|, were the following:

\begin{itemize}
	\item A new deep convolutional neural network for fast hyperspectral image classification: 123 citations.  \cite{cnn-hs-class}
	\item Automatic ship detection in remote sensing images from google earth of complex scenes based on multiscale rotation Dense Feature Pyramid Networks: 97 citations. \cite{ship-rdfpn}
	\item Deep learning in remote sensing applications: A meta-analysis and review: 93 citations. \cite{dl-remote-sensing-review}
	\item Building extraction in very high resolution remote sensing imagery using deep learning and guided filters: 92 citations. \cite{vhr-building}
	\item Very Deep Convolutional Neural Networks for Complex Land Cover Mapping Using Multispectral Remote Sensing Imagery: 72 citations. \cite{very-deep-cnn-lc}
\end{itemize}

Four results were filtered out. \cite{dl-for-cv} with 251 citations, \cite{dl-lungs} with 87 citations, \cite{maoxian-landslide} with 74 citations, and \cite{state-of-the-art-dl} with 73 citations. They were filtered out only because they do not correspond with the~main focus of the~thesis on the classification task with object detection and semantic and instance segmentation in the field of remote sensing. 

The first article, \textit{A new deep convolutional neural network for fast hyperspectral image classification}, starts the~research again in a very positive way. Authors have compared their own architecture with the \zk{MLP} amd three different \zk{CNN}s - a one-dimensional one, a two-dimensional one and a three-dimensional one. Although a~comparison with some popular architectures that just haven't been used on hyperspectral images or some classical \zk{ML} methods would be also interesting, the used ones are properly sourced to another research on hyperspectral image classification and authors underlined main differences between the proposed model and the ones used for an evaluation. All experiments have been conducted on two datasets differing in the number of bands, pixel size of images, spatial resolution, and also in objects of classifiation. Authors also experiment with different patch sizes and - importantly - with the~number of samples per class.

\textit{Automatic ship detection in remote sensing images from google earth of complex scenes based on multiscale rotation Dense Feature Pyramid Networks} also does not compare the~proposed methodology with basic \zk{ML} approaches, but uses a lot of \zk{CNN} models to compare their architecture with - Faster \zk{R-CNN}, Feature pyramid network (\zk{FPN}) \cite{fpn}, rotation region proposal network (\zk{RRPN}) \cite{rrpn}, and rotational region convolutional neural network (\zk{R2CNN}) \cite{r2cnn}. However, as the~paper focuses only on the~ship detection, there is no experiment on other datasets.

The third most cited article on Scopus - \textit{Deep learning in remote sensing applications: A meta-analysis and review} - is a review coming from similar impulses as this thesis or partly \cite{landslide-evaluation} mentioned in chapter \ref{wos}. The motivation is formulated in the sense that \textit{"it appears that a more systematic (i.e. quantitative) analysis is necessary to get a comprehensive and objective understanding of the applications of DL for remote-sensing analysis."}. Authors focus on many more fields than is the goal of this thesis, including image fusion, image registration, scene classification, object detection, land use and land cover classification, image segmentation, object-based image analysis (\zk{OBIA}), and other tasks. The~value of the~paper lies in its extensive research on what are the~most frequent targets of the~use of deep learning (\zk{DL}) for remote sensing, what are the~most frequent \zk{DL} models, spatial resolutions, application areas (urban, vegetation, etc.), average accuracies, common training datasets and even the~most used scientific terms in titles and abstracts of these papers. Although it is a high class review and it works as a valuable overview about \zk{DL} stronger and weaker positions in the~field, authors have not conducted their own experiments, so only results reported in the~original papers are mentioned, if mentioned.

\textit{Building extraction in very high resolution remote sensing imagery using deep learning and guided filters} again proposes a new architecture and compares it to those of SegNet \cite{segnet}, fully convolutional network (\zk{FCN}) \cite{fcn}, a combination of \zk{CNN} and \zk{RF}, Multi-scale deep network \cite{hierarchical-labeling}, and a combination of \zk{CNN}, \zk{RF}, and conditional random fields (\zk{CRF}) \cite{hierarchical-labeling}, but authors do not explain how exactly are these models built, so it does not give the~reader any solid comparison. Especially terms like \zk{CNN} and \zk{FCN} are so general that they do not imply anything more than the basic approach. Time and memory needs are completely ignored in the~study and even though authors experimented on two datasets, both of them were of German cities and therefore very similar in the~content, although differing in used bands.

% Experimental results demonstrated that our methods were better than the other methods

The last reviewed article - \textit{Very Deep Convolutional Neural Networks for Complex Land Cover Mapping Using Multispectral Remote Sensing Imagery} - deals with land cover mapping using \zk{CNN}s with a focus on wetlands detection. Therefore it is not a surprise that they do not experiment with different datasets; however, it would give much more general overview of \zk{CNN} options in the wetland detection to use data from different parts of the world and not only from Newfoundland and Labrador, Canada. Besides that, the paper experiments with DenseNet-121 \cite{densenet}, Inception V3 \cite{inception}, VGG-16 \cite{vgg}, VGG-19, Xception \cite{xception}, \zk{ResNet}-50, and Inception \zk{ResNet} V2 \cite{inception-resnet}, and also with \zk{ML} methods of \zk{SVM} and \zk{RF}. Authors also use different patch sizes and report processing times.


