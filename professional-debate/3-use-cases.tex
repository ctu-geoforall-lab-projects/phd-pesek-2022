\chapter{Use cases}
\label{use-cases}

The best comparison is the one using things readers already know. The best comparison is the one from the real world and praxis.

% TODO: After the professional debate, delete the note about it
As the aim of this thesis is to evaluate the performance of different \zk{CNN} models on different tasks from the field of remote sensing and make an order in \textit{why is an~architecture X used on Y}, use cases chosen for experiments should correspond with tasks normal remote sensing researchers could face. This chapter presents a~research on these chosen use cases (currently - for the purpose of the professional debate - the phrase \textit{use cases} should be actually used in a singular form as there is only one use case so far).

\section{Urban vegetation detection}
\label{urban-green}

If we simplify the picture of the world to an absolute minimum, we can say that there are only two classes - urban areas and nature (or also water, if not counting it in the nature class). A big part of the nature section is vegetation. Therefore, it is not a surprise that the vegetation detection, as well as the urban areas detection, constitutes a big part of remote sensing tasks. Even if a research does not focus specifically on one of these classes, almost any land cover classification that deals with multi-class continental remote sensing images has one or more classes dedicated to the vegetation, and very probably there will be also a need to recognize urban areas.

One of the most used land cover datasets in Europe is the Coordination of information on the environment (\zk{CORINE}) land cover (\zk{CLC}) inventory\footnote{\url{https://land.copernicus.eu/pan-european/corine-land-cover}} created by the European Environment Agency (\zk{EEA}). Its first pan-European land cover dataset was published in 1990 and updates were produced in the year 2000 and every six years after. It contains 44 land cover classes, and some of them are uncommon in other datasets. One of them are \textit{artificial, non-agricultural vegetated areas}, especially its sub-class \textit{green urban areas}. According to the \zk{CLC} nomenclature \cite{clc-nomenclature}, this class is applicable for parks inside settlements, with or without public access, ornamental gardens, mansions’ green grounds, botanical and zoological gardens situated inside settlements or in contact-peripheral zone of settlement, city squares with greenery, inner spaces of city blocks, cemeteries with vegetation inside or directly attached to settlements, vegetated areas that can potentially be used for recreational purpose even if it is not their main utilisation, such as woods inside urban fabric; according to the same source, this class therefore includes vegetated areas of parks, lawns, flower beds, bushes, trees, park ponds, lakes, fountains, lanes and paths (paved or non-paved) in parks or other vegetated recreational areas, buildings and service facilities associated to parks and botanical or zoological gardens, small sport grounds and facilities < 25 ha inside city parks.

This use case will consist of experiments with \zk{CNN}s conducted on the task of classification of vegetated areas and multi-class classification on urban scenes, but as the \zk{CLC} is being used in many applications (as of September 2020, a search for \textit{corine} in \zk{WoS} returns 1006 results and in Scopus 1240 results) but the task of urban vegetation-vegetation outside urban areas automatic differentiation is not widely researched (see section \ref{urban-green-situation}), a special focus will be held also on this problem.

\subsection{Previous works}
\label{urban-green-situation}

% TODO: Move the info about search engines to the general chapter intro after having more use cases
For research on other works dealing with the task of urban green detection or classification from remote sensing data using \zk{CNN}s, the same accademic search engines as in chapter \ref{motivation} were used - \zk{WoS} and Scopus. Terms \textit{urban green}, \textit{urban vegetation}, and - although they are not the only aim of this use case, they present a big part of urban greens - \textit{parks} were used in search phrases to maximize the grasp of the research. The research was done in September 2020.

\subsubsection{Web of Science}
\label{urban-green-wos}

The following restrictions were used for the first query in \zk{WoS}:

\begin{itemize}
	\item \verb|Query string: urban green convolutional neural network|. To focus only on articles dealing with the urban green classification using \zk{CNN}s.
	\item \verb|Search field: All Fields|. Due to the small number of results, a more restricted query did not make sense.
	\item \verb|Open Access: DOAJ Gold|. To focus only on articles and papers from sources listed on the~Di\-rectory of Open Access Journals (\zk{DOAJ}).
\end{itemize}

\noindent Only 7 results were found with this query, and 6 of them were filtered out as not dealing with the selected task. Articles dealing with individual tree detection were also filtered out. The filtered results were \cite{tree-detection-uav-cnn} with 7 citations, \cite{cascaded-cnn-trees} with 2 citations, \cite{window-zooming-fruit} with 2 citations, \cite{urban-green-quantification} with 1 citation, \cite{urban-green-obesity} with 0 citations, and \cite{urban-green-contamination} with 0 citations.

The only result dealing with the chosen task was \textit{Assessing alternative methods for unsupervised segmentation of urban vegetation in very high-resolution multispectral aerial imagery} \cite{urban-green-unsupervised-aerial} with 0 citations, although only a bit as the vegetation is in the end urban only because of the chosen area and the difference between urban and rural vegetation is not mentioned. Authors focus on the unsupervised semantic segmentation of a selected urban area, classifying pervious classes water / shadow, soil / ground, shrub / tree, grass, and other. Authors of the article use the National agricultural imagery program (\zk{NAIP}) dataset with the spatial resolution of 0.6 to 1 meters and 4 bands (blue, green, red, and near infrared with the wavelength of 808 to 882 nanometers) enhanced by the \zk{NDVI} and enhanced vegetation index (\zk{EVI}) to augment the vegetation signal, the soil adjusted vegetation index (\zk{SAVI}) and crust index (\zk{CI}) to minimize background soil effects, and atmospherically resistant vegetation index (\zk{ARVI}) and visual atmospheric resistance index (\zk{VARI}) to reduce atmospheric and topographic effects, resulting in 10 bands. To avoid sham classifications, authors firstly mask out pixels presenting impervious classes, a step in which their approach largely differs from the intended goal of this thesis in this task. They compare k-means clustering \cite{k-means} (using k-means++ seeding \cite{k-means-plusplus} to minimize the eventuality of having a wrong initialization), clustering with a Gaussian mixture model (\zk{GMM}) \cite{gmm}, and a \zk{CNN} model based on \cite{cnn-hs-unsupervised-fuzzy}. The article faces a classical problem of the unsupervised learning evaluation about how to evaluate such methods, but using the Davies-Bouldin index (\zk{DBI}) \cite{dbi} and ad-hoc manually labeled data, k-means surprisingly outperforms both \zk{GMM} and \zk{CNN}.

% Myint et al. [8] describe, to classify objects like landscape features, the spatial resolution of the imagery “needs to be at least one-half the diameter of the smallest object.”

The following restrictions were used for the second query in \zk{WoS}:

\begin{itemize}
	\item \verb|Query string: urban vegetation convolutional neural network|. To focus only on articles dealing with the urban vegetation classification using \zk{CNN}s.
	\item \verb|Search field: All Fields|. Due to the small number of results, a more restricted query did not make sense.
	\item \verb|Open Access: DOAJ Gold|. To focus only on articles and papers from sources listed on the~Di\-rectory of Open Access Journals (\zk{DOAJ}).
\end{itemize}

\noindent The top results from the query, when ordered by the~attribute \verb|Cited by| and filtered as will be described below, were the following:

\begin{itemize}
	\item Classification and Segmentation of Satellite Orthoimagery Using Convolutional Neural Networks: 129 citations. \cite{cnn-satellite-orthoimagery}
	\item Learnable Gated Convolutional Neural Network for Semantic Segmentation in Remote-Sensing Images: 3 citations. \cite{gated-cnn-rs}
	\item Dual and Single Polarized SAR Image Classification Using Compact Convolutional Neural Networks: 3 citations. \cite{polarized-sar-cnn}
	\item Assessing alternative methods for unsupervised segmentation of urban vegetation in very high-resolution multispectral aerial imagery: 0 citations. \cite{urban-green-unsupervised-aerial}
\end{itemize}

\noindent Only 9 results were found with this query and 5 of them were filtered out as not dealing with the selected task. The filtered results were \cite{urban-green-trees-worldview} with 12 citations, \cite{cnn-urban-aerial} with 4 citations, \cite{urban-green-quantification} with 1 citation, \cite{dl-vegetation} with 0 citations, and \cite{tibet-ice-ml} with 0 citations. \cite{urban-green-unsupervised-aerial} with 0 citations was reviewed already previously in this section, therefore it is being skipped now.

\textit{Classification and Segmentation of Satellite Orthoimagery Using Convolutional Neural Networks} \cite{cnn-satellite-orthoimagery} with 129 citations deals with a multi-class classification of an urban scene. The five differentiated classes are vegetation, ground, road, building, and water; no difference between urban and rural vegetation is mentioned in the paper. The used dataset consists of multi-band images enhanced by the \zk{DSM}; the choose of spectral bands was done using two wrapper feature selection methods \cite{wrapper-feature-selection}, sequential forward selection (\zk{SFS}) and sequential backward selection (\zk{SBS}). Authors experiment with two \zk{CNN} architectures, a one-layered one and four one-layered ones used in parallel to boost the multiscale feature learning, as proposed in \cite{multiscale-parallel-cnn}; even with such shallow \zk{CNN}s, authors claim to outperform other \zk{ML} methods, and although results of otheer methods are taken from different tasks and datasets, the performance of \zk{CNN}s is still very high (reaching overall accuracy of 94.49 \% for the parallel approach). The biggest problem in this study was detection of buildings in shadows as these were often misclassified as vegetation.

\textit{Learnable Gated Convolutional Neural Network for Semantic Segmentation in Remote-Sensing Images} \cite{gated-cnn-rs} with 3  citations deals with two tasks of urban detection. The first one is a binary building-nonbuilding Massachusetts dataset \cite{massachusetts-dataset} consisting of RGB images with a spatial resolution of 1 meter. The second one is a Potsdam dataset\footnote{\url{http://www2.isprs.org/commissions/comm3/wg4/2d-sem-label-potsdam.html}} with six classes impervious, building, low vegetattion, tree, car, and clutter consisting of RGB bands enhanced by the infrared band, \zk{DSM}, and normalized digital surface model (\zk{NDSM}) with a spatial resolution of 5 centimeres; however, the infrared band and surface models were not used for experiments, as well as the clutter class. All data were augmented with rotation, mirroring and resizing before the model training. As for the model, authors present their own architecture called learnable-gated convolutional neural network (\zk{L-GCNN}) using a gate function with learnable parameters for multiscale feature fusion, implementing this parameterized gate module (\zk{PGM}) also into different encoder levels. Authors achieve very good results, outperforming other researched architectures especially for artificial objets with large variations in visual appearance and size, on the other hand authors admit that the architecture will very probably not perform well for small training datasets.

\textit{Dual and Single Polarized SAR Image Classification Using Compact Convolutional Neural Networks} \cite{polarized-sar-cnn} with 3 citations deals with the task of a single-polarized and dual-polarized synthetic aperture radar \zk{SAR} image classification. Authors propose a small, compact \zk{CNN} architecture consisting of single hidden \zk{CNN} and \zk{MLP} layers; these parameters give the model few advantages over deep architectures, as its operational performance in work with small data, its speed reduction, and the fact that it does not need any initial feature extraction stage. The first benchmark dataset is a single-polarized \zk{SAR} image from the area of Po delta with a spatial resolution of 3 meters and including classes urban fabric, arable land, forest, inland waters, maritime wetlands, and marine waters. The second benchmark dataset is a dual-polarized \zk{SAR} image from the city of Dresden with a spatial resolution of 4 meters and including classes urban fabric, industrial zones, arable land, pastures, forest, and inland waters. In the work with both datasets, the proposed architecture outperformed other mentioned architectures in the most of the experiments (possibly due to the fact that they were not suitable for small datasets and images were upsampled for them) and the architecture was found to perform better when the channels were enhanced with hue, saturation, and intensity (\zk{HSI}) color model and for a sliding window of size $21 \times 21$ to $27 \times 27$.

The following restrictions were used for the last query in \zk{WoS}:

\begin{itemize}
	\item \verb|Query string: parks convolutional neural network|. To focus only on articles dealing with the parks classification using \zk{CNN}s.
	\item \verb|Search field: All Fields|. Due to the small number of results, a more restricted query did not make sense.
	\item \verb|Open Access: DOAJ Gold|. To focus only on articles and papers from sources listed on the~Di\-rectory of Open Access Journals (\zk{DOAJ}).
\end{itemize}

\noindent From the top twenty results obtained with the described query, all were filtered out as not dealing with the selected task. The found results were \cite{dl-rs-built-up-areas} with 17 citations, \cite{dl-bird-detection} with 11 citations, \cite{dnn-iot} with 9 citations, \cite{dl-sl-wetland} with 8 citations, \cite{dl-corneal-epithelium} with 8 citations, \cite{dl-age-estimation} with 7 citations, \cite{cnn-parking-occupacy} with 3 citations, \cite{ground-measurement-forest} with 2 citations, \cite{bike-sharing-destination} with 2 citations, \cite{w-net-lc} with 1 citation, \cite{mimo-fmcw-parking} with 1 citation, \cite{urban-green-obesity} with 0 citations, \cite{robust-parking-surveillance} with 0 citations, \cite{social-media-open-space} with 0 citations, \cite{dcnn-parking-detection} with 0 citations, \cite{review-crowd-monitoring} with 0 citations, \cite{dl-galapagos-snake} with 0 citations, \cite{deep-binary-vehicle} with 0 citations, \cite{cnn-parrots} with 0 citations, and \cite{indoor-positioning-error} with 0 citations.

\subsubsection{Scopus}
\label{urban-green-scopus}